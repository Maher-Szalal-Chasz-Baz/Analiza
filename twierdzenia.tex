\documentclass{article}

\usepackage[utf8]{inputenc}
\usepackage{geometry} \geometry{margin=90pt}
\usepackage[polish]{babel}
\usepackage{polski}
\usepackage{amsmath}
\usepackage{amsthm}
\usepackage{amsfonts}
\usepackage{mathtools}
\usepackage{fancyhdr}
\usepackage{enumitem}

%%%%%%%%%%%%%%%
%   KOMENDY   %
%%%%%%%%%%%%%%%

\newcommand{\R}{\mathbb{R}}
\newcommand{\N}{\mathbb{N}}
\newcommand{\Q}{\mathbb{Q}}
\newcommand{\li}{\lim_{n\to\infty}}
\newcommand{\lis}{\limsup_{n\to\infty}}
\newcommand{\lii}{\liminf_{n\to\infty}}
\DeclarePairedDelimiter\set\{\}

%%%%%%%%%%%%%%%%%%%%%%%%%%%%%%%%%%%%%%%
%   TWIERDZENIA, LEMATY I DEFINICJE   %
%%%%%%%%%%%%%%%%%%%%%%%%%%%%%%%%%%%%%%%

\newtheorem*{arch}{Aksjomat Archimedesa}
\newtheorem*{przenik}{Twierdzenie o przenikających się ciągach}
\newtheorem*{fekete}{Lemat subaddytywny Fekete}
\newtheorem{lemma}{Lemat}
\newtheorem*{wn}{Wniosek}
\newtheorem*{definition}{Definicja}
\newtheorem{theorem}{Twierdzenie}

%%%%%%%%%%%%%%%%%
\begin{document}%
%%%%%%%%%%%%%%%%%    

%%%%%%%%%%%%%%%%%%%%%%%%%%%%%%%%%%%%%%%%%%%%%%%%%%%%%%%%%%%%%%%%%%%%%%%%%%%%%%%%%%%%%%%%%%%%%%%%%%%%%%%%%%%%%%%%%
                                            \section{Ćwiczenia 7.10.2022}
%%%%%%%%%%%%%%%%%%%%%%%%%%%%%%%%%%%%%%%%%%%%%%%%%%%%%%%%%%%%%%%%%%%%%%%%%%%%%%%%%%%%%%%%%%%%%%%%%%%%%%%%%%%%%%%%%

\begin{arch} %Archimedes
Dla każdej liczby rzeczywistej $a$ istnieje liczba naturalna $n$ taka, że $n>a$.
    \begin{equation*}
        \forall_{a\in\R}\exists_{n\in\N}\enspace n>a
    \end{equation*}
\end{arch}

\begin{lemma} %Supremum
Niech $A\subseteq\R$ będzie zbiorem ograniczonym z góry, a M pewnym ograniczeniem górnym zbioru A.
Wówczas równoważne są zdania:
    \begin{enumerate}[label=(\roman*)]
        \item $M=\sup{A}$;
        \item $\forall_{\varepsilon>0}\exists_{a\in A}\enspace a>M-\varepsilon$.
    \end{enumerate}
\end{lemma}

\begin{lemma} %Infimum
Niech $A\subseteq\R$ będzie zbiorem ograniczonym z dołu, a m pewnym ograniczeniem dolnym zbioru A.
Wówczas równoważne są zdania:
    \begin{enumerate}[label=(\roman*)]
        \item $m=\inf{A}$;
        \item $\forall_{\varepsilon>0}\exists_{a\in A}\enspace a<m+\varepsilon$.
    \end{enumerate}
\end{lemma}

\noindent \textbf{Wnioski.}
    \begin{enumerate}[label=(\Roman*)]
        \item Jeśli $0<a<b$. to istnieje $n\in\N$ takie, że $n\cdot a>b$.
        \item Jeśli $x_1<x_2$, to istnieją $\varepsilon_1,\varepsilon_2>0$ takie, że $(x_1-\varepsilon_1,x_1+\varepsilon_1)\cap(x_2-\varepsilon_2,x_2+\varepsilon_2)=\emptyset$
        \item $\forall_{h>0,a\in\R}\exists_{k\in\N} \enspace (k-1)h\le a<kh$
        \item $\forall_{a<b}\forall_{a,b\in\R}\exists_{c\in\R}\exists_{c'\in\R\setminus\Q} \enspace c,c'\in(a,b)$
    \end{enumerate}

\begin{lemma} %supremum sumy i supremum różnicy
    Dla dowolnych podzbiorów $A,B \subseteq\R$ mamy
        \begin{enumerate}[label=(\alph*)]
            \item $\sup{(A+B)}=\sup{A}+\sup{B}$,
            \item $\sup{(A-B)}=\sup{A}-\inf{B}$.
        \end{enumerate}
\end{lemma}

\begin{lemma} %supremum -A
    Dla niepustego podzbioru $A\subseteq$ mamy $\sup{(-A)}=-\inf{A}$
\end{lemma}

\begin{definition} %Ciąg zstępujący
O ciągu zbiorów $A_1, A_2,\ldots$ mówimy, że jest zstępujący, jeśli $A_1\supseteq a_2 \supseteq \ldots$
\end{definition}


\begin{theorem} %Cantor
    Jeśli $I_1, I_2, \ldots$ jest ciągiem zstępujących przedziałów domnkniętych prostej rzeczywistej
    to $\bigcap_{n\in\N}A_n\ne\emptyset$.
\end{theorem}

%%%%%%%%%%%%%%%%%%%%%%%%%%%%%%%%%%%%%%%%%%%%%%%%%%%%%%%%%%%%%%%%%%%%%%%%%%%%%%%%%%%%%%%%%%%%%%%%%%%%%%%%%%%%%%%%%
                                            \section{Ćwiczenia 21.10}
%%%%%%%%%%%%%%%%%%%%%%%%%%%%%%%%%%%%%%%%%%%%%%%%%%%%%%%%%%%%%%%%%%%%%%%%%%%%%%%%%%%%%%%%%%%%%%%%%%%%%%%%%%%%%%%%%

\begin{lemma}
    Dane są dwa ciągi $(x_n)$, $(y_n)$. Jeśli $(y_n)$ jest ograniczony oraz $x_n\to 0$, wówczas
    $x_n y_n \to 0$.
\end{lemma}

\begin{theorem}
    Jeśli $a_n\to g$, to $A(a_1,a_2,\ldots,a_n)=\frac{a_1+a_2+\ldots+a_n}{n}\to g$.
    \end{theorem}    

%%%%%%%%%%%%%%%%%%%%%%%%%%%%%%%%%%%%%%%%%%%%%%%%%%%%%%%%%%%%%%%%%%%%%%%%%%%%%%%%%%%%%%%%%%%%%%%%%%%%%%%%%%%%%%%%%
                                            \section{Ćwiczenia 25.10}
%%%%%%%%%%%%%%%%%%%%%%%%%%%%%%%%%%%%%%%%%%%%%%%%%%%%%%%%%%%%%%%%%%%%%%%%%%%%%%%%%%%%%%%%%%%%%%%%%%%%%%%%%%%%%%%%%

\begin{przenik}
Niech dane będą trzy ciągi liczb rzeczywistych $(x_n), (y_n)$ oraz $(c_n)$. Jeśli $x_n\to x$ i 
$y_n\to y$ oraz $c_n=\frac{x_1y_n+\ldots+x_ny_1}{n}$, wówczas $c_n\to xy$.
\end{przenik}

%%%%%%%%%%%%%%%%%%%%%%%%%%%%%%%%%%%%%%%%%%%%%%%%%%%%%%%%%%%%%%%%%%%%%%%%%%%%%%%%%%%%%%%%%%%%%%%%%%%%%%%%%%%%%%%%%
                                            \section{Ćwiczenia 28.10.2022}
%%%%%%%%%%%%%%%%%%%%%%%%%%%%%%%%%%%%%%%%%%%%%%%%%%%%%%%%%%%%%%%%%%%%%%%%%%%%%%%%%%%%%%%%%%%%%%%%%%%%%%%%%%%%%%%%%

\begin{lemma} %Zbiór podciągów zbieżnych
Niech $S=\set{\lim_{n\to\infty}x_n:(x_{n_k})-\text{podciąg zbieżny, ciągu $(x_n)$, do granicy 
skończonej lub nie}}$. Jeśli $+\infty,-\infty\notin S$, to zbiór S jest ograniczony, więc zawiera 
podciąg zbieżny, zatem $S\ne \emptyset$.
\end{lemma}

\begin{lemma} %Gdy -infty i +infty nie należą do S
Przyjmując oznaczenia jak powyżej:
    \begin{enumerate}[label=(\alph*)]
        \item Jeśli $-\infty\notin S$, to $\inf S\in\R$ lub $\inf S=+\infty$,
        \item Jeśli $+\infty\notin S$, to $\sup S\in\R$ lub $\sup S=-\infty$.
    \end{enumerate}
\end{lemma}

\begin{lemma} %ciąg ograniczony i sup S, inf S
    Jeśli $(a_n)$ jest ograniczony z góry/dołu, to $\sup S/\inf S\in S$.    
\end{lemma}

\begin{definition} %granica górna i dolna
    \begin{align*}
        \text{Granicą górną ciągu $(x_n)$ nazywamy}\enspace \lis{x_n}\stackrel{def}{=}\sup{S}.\\
        \text{Granicą dolną ciągu $(x_n)$ nazywamy}\enspace \lii{x_n}\stackrel{def}{=}\inf{S}.
    \end{align*}
\end{definition}

\begin{fekete} %Fekete
Jeśli ciąg $(x_n)$ spełnia warunek: $x_{n+m}\le x_n+x_m,$
to istnieje granica $\li\frac{x_n}{n}$ i co więcej
    \begin{equation*}
        \li\frac{x_n}{n}=\inf\set{\frac{x_n}{n},n\in\N}.
    \end{equation*}
\end{fekete}

%%%%%%%%%%%%%%%
\end{document}%
%%%%%%%%%%%%%%%